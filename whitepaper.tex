\documentclass{article}
\usepackage{multicol}
\author{Authors}
\title{NFT-L : A decentralised, verifiable protocol for skill credentialing}
\begin{document}
  \maketitle
  \begin{abstract}
    Traditionally, we've used certificates issued by centralised authorities as proof of possessing a skill.
    This makes the certificates vulnerable to hampering, and makes it very hard to audit the issuance in the first place.
    We're introducing NFT-L. A non fungible token for learning. These are non transferable tokens of certification of a skill that are decentralised and auditable.
    These are implemented and currently live on QuestBook - a learning app on iOS and Android for professionals. All these NFT-Ls live outside the app itself in the form of an \texttt{EIP 1268} token on the Etherium main-chain.
    This whitepaper introduces the semantics of the NFT-L, Creator Coins for each user on the platform, their issuance algorithms and an enabling ERC-20 token called Learn Coins.
  \end{abstract}
  \section{Introduction}
    Today the most popular ways of reporting that an individual posses a skill is either through a certificate issued by a certification programme or, worse, by self reporting them on social media.
    Here we highlight some of the problems in the traditional mode of skill development and assessment. 
    \subsection{Problems in - Incentives for teachers}
      Most teachers are teaching on platforms like Teachable, Kajabi and Udemy - in the hope that people will pay them for what they are teaching. 
      This means that the teacher cares only about the number of people coming into the course. The quality of course delivery and making an impact on the learners lives seems to be a distant secondary metric. Teaching is no longer about learning outcomes. It's turned into a popularity contest of attracting largest number of enrolments. 
   
      The impact is probably captured, at best if at all, through an easy to game 5 point rating system or conveniently ranked comments section.
      \break
       Online education market places are centralised and reward teachers for enrolments. We think this is fundamentally flawed and incentive structures are designed for a poor learning experience. Using crypto we can create a far better incentive structures that are directly tied to learning outcomes. We propose teachers issuing tokens. This creates a far higher liquidity. For example, if Acme Corp wants to hire graduates from a teacher's courses, they will invest in her. Or the alumni who found the course useful will buy the tokens because they know it'll be useful for more.
       
       With NFT-L, teachers will earn more without charging students. Through small classes and ensuring learning outcomes.
     
       
       \subsection{Problems in - Incentives for learners}
      The incentive for a learner is rarely the learning itself. Apart from a handful, most people want to be able to make an impact in their career and their economic mobility by virtue of learning something new. 
      Most certification programmes focus on being able to produce a correct answer in an examination. The incentives of the learner should be tied such that they are encouraged to participate in discussions, make mistakes and collaborate to improve their skills.
      \break
      In the proposed solution, the incentives are so structured that \textbf{it is free to learn - but expensive to not learn}. Because we provide a learning mode that involves close collaboration with an expert in the field, being respectful of their time is critical. 
      We want to consciously weed out people who are here only for information porn and not real learning.
      With NFT-L, learning is always free. Learners learn along with a small cohort of motivated peers interested in the same thing. With the guidance of an expert teacher.
      \subsection{Problems in - Certification Verifiability}
      \subsubsection{Contacting the issuing authority}
        The only way to check if a certificate has not been tampered with or created fraudulently, one must contact the claimed issuing authority to verify if they actually issued the certificate. 
        This process is usually non-automatable and cumbersome that most people end up giving the benefit of doubt to the applicant. 
        \break
        We will be proposing an on-chain certificate that will be verifiable and auditable in real-time.
      \subsubsection{No history}
        Most certification programs happen behind closed doors. If an individual produces a certificate it is impossible to know how the certificate was earned. There are probably no records, or even if they are - they aren't publicly viewable. 
        \break
        Now that most of the learning is now happening online over chat platforms like Slack, Discord, Telegram and Zoom, it is possible to have a record of the \textit{trail of learning}. In the solution we will be proposing, every skill certificate will be auditable using publicly available information.
  \section{Specifications}
    Questbook is a \textit{specification by implementation} of the proposed learning standards.
    \subsection{New format of learning - Quests}
      We've traditionally been used to only a big-bang learning approach either in the form of years long or months long courses.
      On Questbook, every course is to be broken down into small bite-sized courses called Quests that are both atomic and independent.
      A Quest consists of 
      \begin{itemize}
        \item \textbf{A Questmaster} who will run the Quest
        \item \textbf{A Questroom} where the participants can chat with each other and the Questmaster. At any point in time a Quest room can only have  a fixed number of participants - thereby, limiting the number of people who can participate in the Quest at any point of time.
        \item \textbf{Quest material} is what the participants must read/watch before participating in the Quest room. These could be videos, blogs or podacasts that are required reading to complete a task in the Quest.
        \item \textbf{Quest task} is what the participants must apply the learnings on. This is what the Questmaster will evaluate the participant on. E.g. \textit{Write a for loops in Python to draw a triangle}
        \item \textbf{A proof of learning} at the end of the Quest. A Questmaster approves a participant when they have mastered the material shared in the Quest. Once approved a non-fungible non-transferable token (certificate) is credited to the participant's wallet
      \end{itemize}
    \subsection{New Certification - NFT-L}
      We propose NFT-Ls as a global standard in issuing certificates against attaining a skill.
      \subsubsection{Public Key}
        Every NFT-L (non-fungible token of learning), will be issued on the main Ethereum chain and queryable by the publickey to get further details about the issuance.
      \subsubsection{Issued By}
        This is the public key of the \textbf{Questmaster} who ran the Quest. Who ran the Quest is a very important signal in who issued the Quest. We will see later that the incentives of the Questmaster are tightly tied to the quality of learning accrued by participants in the Quest. 
      \subsubsection{Issued To}
        The public key to whom the NFT-L has been issued. This is, unlike regular NFTs, \textbf{non-transferable}. 
      \subsubsection{Quest information}
        The Quest that this NFT-L was issued as a part of. Each Questmaster could potentially have multiple Quests running. 
      \subsubsection{Time to complete}
        How long it took for the participant to complete the Quest. Though not fool-proof, someone learning something faster than the others assuming they all started at a level ground is a valid signal in the quality of learning of an individual. 
      \subsubsection{Metadata}
        \begin{itemize}
          \item Date of issuance
          \item NFT-L Merkel root
          \item Signature of issuing platform (e.g. Questbook)
        \end{itemize}
    \subsection{New Creator Economy Standard - Creator Coins}
      Every teacher has their own unique set of Creator Coins. This will be based on an ERC-20 Token standard, minted by a Proof of Authority source.
      \subsubsection{Genesis}
        Every teacher gets, upon sign up, 10 Creator Coins with their name on it. e.g. Alice Creator Coin, Bob Creator Coin.
      \subsubsection{Price}
        There is a price to pay in Learn Coins to mint new Creator Coins. The cost of minting a Creator Coin keeps going up exponentially.
        Number of Creator Coins of Alice you get \(\eta_{alice}\) by paying 1 Learn Coin is dependent on how many Alice Creator Coins have been minted (\(N_{alice}\)) 
        \[
          \eta_{alice} = 10^{-(0.03 * N_{alice})}
        \]
        The price paid goes directly to the Creator.
      \subsubsection{Scarcity}
        There is no theoretical limit on the number of Creator Coins of a particular Creator. However with growing number of Creator Coins in circulation, the prices will get prohibitively high.
      \subsubsection{Specifications}
        Extension of an ERC-20 token specification.
        \begin{itemize}
          \item Creator Public Key
          \item Owner Public Key
          \item Issuing Authority
          \item Value (number of coins)
        \end{itemize}

      
  \section{Teaching Process}
    \subsection{Gating the quality of Quest creators}
      Only high quality Quest creators will be allowed on to the platform to create Quests. To create Quests, the teacher must list their profile along with what they will teach in Quests on a public voting platform.
      Companies and individuals who want to see that Quest becoming a reality will vote on the Quests by paying up Learn Coins. The top \textit{n} voted quest creators will get to start quests every 24 hours. \textit{n} will double every month
      \[
        n = 2^{t_{month}}
      \]
      This ensures only good quality and respected creators are actually dropping Quests. Also means that there is sufficient demand for the skill being imparted by this Quest that could potentially lead to an increase in economic mobility.
      All the Learn Coins collected in the voting process is credited to the creator.
    \subsection{Cohort size}
      Questbook recommends Questmasters start their first Quest with a small group of four participants. Gradually increasing the cohort size with experience. 
      This is because it is hard enough to teach a group of learners, but even more so on a new platform.
    \subsection{Gating participants}
      All the users who have shown interest in parcipating in the Quest are listed for the Questmaster to approve entry to the Quest. They are prioritized based on the following two signals.
      \subsubsection{Staking an NFT-L}
        The Questmaster can optionally define a pre-requisite NFT-L for every Quest. That means, people who have completed the other Quest will get an advantage in the queue.
        This is also a way to tell the participants as to what knowledge is required to participate in this Quest to get the maximum benefits. 
      \subsubsection{Staking Learn Coins}
        If there aren't enough applicants who already own a pre-requisite NFT-L, the other participants are sorted by how many Learn Coins they are willing to stake to complete the Quest.
        We will see later that more the participant stakes, more they are invested in learning what is being shared in the Quest.
    \subsection{Time limit}
      The Questmaster also sets their availability by setting a tentative duration to the Quest. The participants must complete the Quest before the specified time. Once the Quest is ended by the Questmaster, the participants can no longer get an NFT-L in this version of the Quest.
      Again, this is to be respectful of the experts running these Quests. 
    \subsection{Quest Edition}
      The Questmaster can choose to run multiple editions one after another of the same Quest. Each edition opens up the voting mechanism again, where institutes and individuals pay the Questmaster for the new edition. This is how the Questmaster primarily earns on this platform.
  \section{Learning Process}
    \subsection{Choosing a Quest}
        There are various signals that will be made public to the users who are browsing through all the available Quests. We leave it to the open market dynamics to pick the winning Quests.
        \begin{itemize}
          \item \textbf{Questmaster Reputation }: Reputation of the Questmaster is a function of how many participants participated in all their Quests aggregated and the number of NFT-Ls they've issued. Indicating the quality of the Questmaster.
          \item \textbf{Quest edition }: How many times has this Quest been run before. More it has been run, more it has been battle tested.
          \item \textbf{Votes }: Number of Learn Coins paid to Questmaster to run this Quest - thereby suggesting the market value of the skill taught in this Quest. Resets to zero at every edition of the Quest.
          \item \textbf{Avg. time to complete Quest }: Those who have been awarded an NFT-L in an earlier edition of this Quest, how long did it take them to get it. Indication of time commitment required from the learner. 
          \item \textbf{Last issued NFT-L }:  When was the last NFT-L issued in any earlier or current edition of this Quest. This shows how active the Quest is right now. 
          \item \textbf{Pre-requisite NFT-L }: An NFT-L that is suggested pre-requisite to attend this Quest. Suggests what knowledge is required to make the most off of this Quest.
        \end{itemize}
    \subsection{Learning Material}
        The learning material is public and anyone can view it without participating in the Quest. 
        These are typically links, videos or PDFs. The user can see if they want to master the material by participating in the Quest. 
        Learning Material is only a small part of the learning process. Lot of people are used to assimilating information but never applying it. 
        Anyone is free to open these learning material. The ones who want to master can choose to participate in the Quest by requesting an entry to the Quest.
    \subsection{Requesting Entry}
        All the Quests are gated and number of participants limited. To participate, one must produce one of :
        \subsubsection{A pre-requisite NFT-L}
          If the learner owns the pre-requisite NFT-L they are given automatic entry to the Quest on a first come first serve basis. The NFT-L is automatically staked in the Quest.
        \subsubsection{Staking Learn Coins}
          If the learner doesn't have the pre-requisite NFT-L and wants to participate, they can stake some Learn Coins.
    \subsection{Completing a Quest}
        \subsubsection{Public Chat}
          The Quest can be run in any format the Questmaster deems appropriate, by coordinating with the participants directly over a public chat. 
          Questbook recommends the Questmasters conduct all the communications in the public. Questbook automatically archives all the chats in a public repository.
          This public repository can be accessed whenever someone wants to audit a particular NFT-L.
        \subsubsection{Request for NFT-L}
          The issuance of the NFT-L is initiated by the learner. The learner uploads a \textit{Request for NFT-L} as a special attachment in the Questroom chat.
        \subsubsection{Approval or Rejection of the NFT-L}
          The Questmaster can then look at the \textit{Request for NFT-L} and either accept it or reject it along with a voice/text that stays as an authentic commentary with the NFT-L forever, if approved. 
          If rejected, the chat continues and the learner can attempt a request any number of times.
        \subsubsection{Recovering the Stake}
          If the NFT-L is issued, the participants gets back, along with the new NFT-L, the staked NFT-L or the staked Learn Coins.
          \break
          If the participant is not able to get an NFT-L before the Quest ends, the staked NFT-L is burnt and the staked Learn Coins goes into a community pool.
  \section{Voting process}
    Voting for new Quests and re-runs of existing Quests is a core pillar of this economy. 
    There are few reasons for the voting
    \begin{itemize}
      \item Shows market demand for the learning outcomes - individuals or institutions voting will be the ones who want to see the material taught and more people skilled in that particular field.
      \item This is also how the Questmaster is incentivized monetarily. We don't think want to charge the learners for learning.
      \item Bragging rights!
    \end{itemize}
    They can purchase Creator Coins from the creator or from a public exchange. If they buy it from the Creator by voting, they are also incentivizing the creator to actually run the Quest.
    If the price is prohibitively high because of the pricing function, they can buy a small token amount from the Creator to incentivize them and buy more tokens from a public exchange to rank high in ownership of that Creator Coins.
  \section{Uses of NFT-L}
    It is uncertain as to where all NFT-Ls will be used, but it is clear that the application will be as wide, if not wider, as the expanse of certificates itself.
    \subsection{Job Opportunities}
      The institutions who had paid the Questmaster for a particular Quest, could have had the motive of building skills that they want to hire. Producing such an NFT-L to such institutes will be a great optimization in their hiring process.
    \subsection{Grants and Scholarships}
      Questbook will be working closely with institutions and organizations to give out grants like Gitcoin Grants or Scholarships from governments or other NGOs.
  \section{Uses of Creator Coins}
    To get the benefits of holding a Creator Coin, the user must own atleast 1.0 Creator Coins or more. 
    \begin{itemize}
      \item Full history of chat when the Quest is on-going, for all Quests of that Questmaster. Others will be able to see only the latest messages till the Quest ends. All the chats are archived and made public only after the end of the Quest.
      \item Priority intimation of NFT-L issuance. The NFT-L issuance is made public information only 168 hours after the actual issuance. The issuance notification is sent to the Questmaster coin holders in order of priority defined by the number of Coins of that Questmaster held by the user.
      \item Trade it for on a public exchange
    \end{itemize}

  \section{Initial incentivization Model}
    To bootstrap the initial liquidity on the platfrom we will be incentivizing the early customers with Learn Coins - based on a definitive logic.
    First 3M Learn Coins will be distributed as a part of the Initial incentivization plan.
    \subsection{Questmasters initial Incentive Plan}
      Total pool = 2M Learn Coins.
      The factors that define the number of Learn Coins a Questmaster \(L\) gets on dropping the first Quest :
      \begin{itemize}
        \item Sum of \textbf{Number of Coins in votes}, \textit{v}
        \item \textbf{Authority score} on other social media, \textit{s}
        \item \textbf{Editors' score} based on who will be most useful to the platform as decided by a jury, \textit{e}
      \end{itemize} 
      \[
        L = \Sigma v + (S * s) + e
      \]
      \(S = 0.002\)

    \subsection{Learners Initial Incentive Plan}
      Total Pool = 1M Learn Coins. The factors that define number of coins a learner gets upon completion of their first Quest \(L\)
      \begin{itemize}
        \item \textbf{Authority score} on other social media, \textit{s}
        \item \textbf{Time to acceptance} into the first applied Quest, \(t_{accept}\) (number of hours)
        \item \textbf{Time to complete Quest}, \(t_{complete}\) (number of hours)
        \item \textbf{Editor's score} based on who will be most useful to the platform as decided by a jury, \textit{e}
      \end{itemize}
      \[
        L = \frac{S * s}{10 * (5*t_{accept} + t_{complete}) + e}
      \]
    \subsection{Ongoing incentivization}
      Every month 100,000 Learn Coins will be distributed to the top \(5\%\) of the Questmasters using the following weighting \(w\) 
      \subsubsection{Questmaster}
        \begin{itemize}
          \item Number of NFT-Ls issued ever by this Questmaster, \(N\)
          \item Number of Learn Coins paid as vote in the last completed Quest \(v_{last}\)
          \item Number of NFT-Ls issued by this Questmaster staked and recovered ever \(R\)
          \item Number of Learn Coins staked in last ended Quest \(\sigma_{last}\)
          \item Number of months since the ending of the last Quest \(t_{last}\)
        \end{itemize}
        Equation for distributing Learn Coins to Questmasters :
        \[
          w =  v_{last} * e * \frac{R}{N} * \frac{t_{last}^{0.9}}{e^t_{last}}
        \]

      \subsubsection{Learner }  
        Every month 100,000 Learn Coins will be distributed to the top \(20\%\) of the Learners using the following weighting \(w\) 
        \begin{itemize}
          \item NFT-Ls earned in Quest \(q\)
          \item Total Learn Coins used to vote on the Quest \(\sigma_q\)
          \item Total NFT-Ls issued so far in the Quest \(N_q\)
        \end{itemize}
        \[
          w = \Sigma_{q} \frac{\sigma_q}{N_q}
        \]
  \section{Governance}
    \subsection{Issuance of NFT-Ls}
      The NFT-L is an open standard that can be issued by any individual. However, the NFT-L also holds an optional Meta-data that signifies which platform this NFT-L was issued on.
      Any platform can implement the Questbook specification or improve upon it and implement their own specification. That signature when added to the meta-data assigns some credibility as to how the NFT-L was issued.
      \break
      There is no requirement for an NFT-L to be issued on the Questbook app alone. Infact, we would encourage more platforms to adopt and improve a standard for learning certification that is decentralized and verifiable.
    \subsection{Learn Coins}
      Learn Coins will be minted in a Proof-of-Authority model on the Questbook App. Every month 200,000 Learn Coins will be minted. The damping function will be decided on a later date.
  \section{Example}
    Here we will illustrate end to end how the ecosystem will use the specifications and implementation.
    \par
    Acme Corp is a company that builds and sells subscription to a CRM software.
    \par
    Alice used to work at Acme Corp but recently started her own consulting firm to setup CRMs for companies. She runs a popular Youtube channel too about CRMs.
    \par
    Bob is a final year MBA student who wants to learn tools that are used in the industry.
    \par
    Charlie is a customer success executive who lost a job due to the pandemic and is looking to upskill himself to earn a new job.
    \par
    Danny is a freelancer who works with CRMs once in a while
    \par
    Acme Corp wants Alice to run Quests that it can hire as customer success executives at Acme Corp.
    Acme Corp usually pays 10k USD to a hiring agency for every customer success hire they make. But they decide to put in 5,000 USD into finding a hire through Questbook. They buy 5,000 USD worth of learn coins and vote for Alice's Quest on \textit{Pro-User of Acme CRM}
    \par
    Alice typically makes 10k USD per month in her consulting firm. So, for her 5K is a welcome top up. She agrees to do the Quest for Acme Corp. She decides to do a series of 5 Quests.
    They are : \textit{Introduction to CRMs, Installing Acme CRM, Importing all customers to Acme CRM, Tips and tricks for Acme CRM, Pro User of Acme CRM}.
    The last one is what Acme Corp paid for, but Alice realized that to do a good job, she needs to take learners through all the hoops. She launches these Quests, where the pre-requisite for each quest is the Quest before it in the list above.
    There is no pre-requisite for the first Quest.
    \par
    Bob joins the first Quest and Alice happily shares what she knows about CRMs by sharing a few of her own Youtube videos. Bob is able to quickly grasp the basics and Alice awards him the NFT-L.
    \par
    When the second Quest starts, Bob having the NFT-L of the previous Quest gets automatic entry to the Quest.
    However Charlie also wants to join this Quest. He didn't want to join the Introduction Quest because he is already experienced in CRMs, just not Acme CRM.
    So, he stakes 100USD worth of Learn Coins and tells Alice why he didn't join the first Quest. Alice sees his point and allows him into the second Quest. 
    Both of them complete the Quest, Bob gets his previous NFT-L back and Charlie his 100USD worth of Learn Coins.
    \par
    Both Bob and Charlie move on to Quest three - \textit{Importing customers}. This time Danny shows up staking only 10 USD worth of Learn Coins. Alice is skeptical, but still lets him in.
    Bob and Charlie complete the tasks that need to be done, but Danny slacks off. Alice ends the Quest, gives Bob and Charlie the NFT-L and Danny's 10 USD worth of learn coins are sent to the community pool. He doesn't get it back. 
    \par
    Again Bob and Charlie glide into the most interesting Quest \textit{Tips and Tricks}. This time Danny shows up again, staking 5 USD this time. 
    Now, Alice knows Danny is just shopping around not serious about learning. Because if he were, he wouldn't have slacked in the previous Quest and would have put in a sizable stake to show seriousness.
    So only Bob and Charlie participate in the Quest and complete it to Alice's statisfaction. 
    \par 
    Both Bob and Charlie have now become good friends with each other and with Alice. They casually move into the most important Quest - Pro user. 
    This time Danny doesn't even show up. Everyone's happy. Bob and Charlie dig deep into all the wisdom Alice has to share and really become Pro users of Acme CRM. Alice, ofcourse grants both of them the NFT-Ls.
    \par
    Acme Corp sees that 2 people have completed the Quests, quickly check how the conversations went in each of the NFT-Ls. They're happy to see how well Alice conducted the sessions. 
    They interview both Bob and Charlie directly with the CEOs not needing the initial rounds. Both of them get hired.
    \par
    Acme Corp sees this is way more efficient than paying a hiring agency. They set out a budget of 50K USD to hire the next 10 customer success executives. They vote for Alice and 2 more Youtubers like her.
    It's a major success.
    \par
    Everyone lives happily ever after. Except Danny, he's now jobless.

      
        

      

\end{document}

