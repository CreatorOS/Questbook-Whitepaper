\documentclass{article}
\usepackage{multicol}
\usepackage{mathtools}

\author{Work in progress}
\title{LearnCoin - Specification for a decentralized learning economy}
\begin{document}
  \maketitle
  \begin{abstract}
    This paper lays out the specifications for a Univeristy DAO.
    This DAO defines a \textit{decentralised learning economy} in which a teacher is economically incentivized only to \textit{drive learning outcomes}.
    \par
    The economy runs  with a native protocol token called  \textit{Learn Coins}, which teachers earn by running courses. 
    Community members pay Learn Coins to incentivize the best teachers to run these courses.
    This economy grows the population of learners by making learning free, accessible and rewarding. %todo : rewarding -> ?
    \par
    This work introduces
    \begin{itemize}
      \item NFT-Ls which are non-transferable tokens of proof of mastery. 
        This is an extension of ERC-721 and EIP-1238, available on the Ethereum main-chain. 
        We will be specifying the open standard for these decentralize certificates.
      \item Verifiable and Auditable proofs of learning. 
        The complete set of events leading to the learning outcomes are publicly stored and accessible on IPFS.
        Every NFT-L references this \textit{trail of learning}
      \item Learn Coins is the underlying currency that is used to vote on what and how the future of learning shapes. 
        We will touch upon the uses of this ERC-20 token multiple times in this paper.
    \end{itemize}
    \par
    These specifications are implemented in Questbook - an app available on iOS and Android.
  \end{abstract}
  \section{Existing issues} 
    Today,
    \begin{itemize}
      %low learning outcome, recruiters not paying, certificates are bogus
      \item A teacher earns by charging a fees from the learner, making affordability a bottleneck to learning. The cost of education is increasing year on year. \cite{feesincrease}
      \item Multiple studies show that learning is more effective in smaller classes \cite{starresearch}. 
        But, the current incentive structures make it economically infeasible to conduct small classes. %TODO(ruchil) - "we need more money" quote
      \item When learning happens, the community benefits. Yet, there is no community coordination to drive learning. %TODO(madhavan) - add crypto citations
      \item Learning happens behind closed doors and there is no audit of the learning process. This is reflected in the elaborate recruitment processes most companies have. \cite{roundsofinterviewcount}
    \end{itemize}

  \section{Solution}
    \subsection{Affordability}
      The learner never has to pay. The community decides what gets taught by backing courses they believe are valuable, thereby paying the teacher.
      \par
      This frees the teacher of having to charge the learners, expanding the universe of learners in the world.
    \subsection{Class sizes}
      The best learning happens in classes of size 1. The smaller the class, more is the mastery attained. If there is \textit{at least one master} produced in a course, the escrow is released and credited to the teacher. There is more value in producing one master, as against producing hundred lackluster candidates.
      \par
      This shifts the incentive model from one of having to run large classes to running small classes and ensuring mastery is attained for atleast one learner. This might eventually stabilize at \textit{each-one-teach-one}.
    \subsection{Community Participation}
      The community incentivizes the teacher by putting in an escrow before the course starts. When the course is completed the community signs off the student who had completed the course to their satisfaction.  
      \par
      Learners will strive to get a sign off from the established community members, and hold that badge proudly. The community benefits because more of their members are becoming masters, not \textit{learning for the sake of learning}.
    \subsection{Auditability}
      We make all the information about what a learner has learnt and how public. This means that anyone can not only know what a person has learnt, but also make it possible to know the trail of events that led to the learning. 
      \par
      This will result in the community not favouring some numeric score, but would incentivize them to look at the work the learner has done during the course, to make a decision about their mastery. 
  \section{Specifications}
    The entire economy is currently being deployed on Questbook - an app available on iOS and Android. In this section we will detail out the specifications of this open standard.
    \subsection{New format of learning - Quests}
      On Questbook, every course is to be broken down into small bite-sized courses called Quests that are both atomic and independent.
      A Quest consists of 
      \begin{itemize}
        \item \textbf{A Questmaster} who will run the Quest
        \item \textbf{A Questroom} where the participants can chat with each other and the Questmaster. At any point in time a Questroom can only have  a fixed number of participants.
        \item \textbf{Quest material} is what the participants must read/watch before participating in the Quest room. These could be videos, blogs or podacasts that are required reading to complete a task in the Quest.
        \item \textbf{A proof of learning} at the end of the Quest. A Questmaster approves a participant when they have mastered the material shared in the Quest. Once approved a non-fungible non-transferable token (certificate) is credited to the participant's wallet
      \end{itemize}
    \subsection{New Certification - NFT-L}
      We propose NFT-Ls as an \textbf{open global standard} as a \textit{proof of mastery}. This will be awarded upon completion of a particular Quest. These NFT-Ls consist of the following information.
      \subsubsection{Public Key}
        Every NFT-L (non-fungible token of learning), will be issued on the main Ethereum chain and queryable by the publickey to get further details about the issuance.
      \subsubsection{Issued By}
        This is the public key of the Questmaster who ran the Quest. The incentives of the Questmaster are tightly tied to the quality of learning accrued by participants in the Quest, as we will see later. 
      \subsubsection{Issued To}
        The public key to whom the NFT-L has been issued. This is, unlike regular NFTs, \textbf{non-transferable}. 
      \subsubsection{Quest information}
        The Quest that this NFT-L was issued as a part of. Each Questmaster could potentially have multiple Quests running. 
      \subsubsection{Metadata}
        \begin{itemize}
          \item Date of issuance
          \item NFT-L Merkel root
          \item Signature of issuing platform (e.g. Questbook)
        \end{itemize}
      
  \section{Teaching Process}
    \subsection{Gating the quality of Quests}
        Only high quality Quests will be allowed on the platform.
        A high quality Quest is one that drives valuable learning outcomes. 
        The community can vote on what gets taught on the platform by staking Learn Coins on individual Quests.
        \par
        The top \(n\) Quests every day will be allowed to start. 
        When a Quest starts, the committed Learn Coins are credited to the Questmaster.
    \subsection{Cohort size}
      Questbook recommends Questmasters start their first Quest with a small group of four participants. Gradually increasing the cohort size. 
      Research shows that smaller class sizes drive better learning outcomes. %TODO(ruchil): add citation
      Questmaster must make the maximum cohort size public before the community is able to start committing.
    \subsection{Gating participants}
      All the users who have shown interest in parcipating in the Quest are listed for the Questmaster to approve entry to the Quest. They are prioritized based on the following.
      \subsubsection{Staking an NFT-L}
        The Questmaster can optionally define a pre-requisite NFT-L for every Quest. That means, people who have completed the pre-requisite Quest will get an advantage - by getting an automated entry to the Quest.
        This tells the participants as to what knowledge is required to participate in this Quest to get the maximum benefits. 
      \subsubsection{Staking Learn Coins}
        If there aren't enough applicants who already own a pre-requisite NFT-L, the other participants are sorted by how many Learn Coins they are willing to stake to complete the Quest. This is not a fees. The stake is returned to the learner when they complete the Quest. However, if they don't complete the Quest, the stake is burnt.
    \subsection{Time limit}
      The Questmaster also sets a tentative duration to the Quest. The participants must complete the Quest before the specified time. 
      Once the Quest is ended by the Questmaster, the participants can no longer get an NFT-L in this version of the Quest. If the participants haven't completed the Quest before the Questmaster ends the Quest, the stake is burnt.
    \subsection{Quest Edition}
      The Questmaster can choose to run multiple editions one after another of the same Quest. Each edition opens up the voting mechanism again, where the community commits fresh Learn Coins on the new edition of this Quest. 
  \section{Learning Process}
    \subsection{Choosing a Quest}
        There are various signals that will be made public to the users who are browsing through all the available Quests. We leave it to the open market dynamics to pick the winning Quests.
        The following are public information for the user to make an informed choice
        \begin{itemize}
          \item \textbf{Questmaster Reputation }: A function of social signals including satisfaction score, community votes, applicability of the NFT-Ls issued by this Questmaster.
          \item \textbf{Quest edition }: How many times has this Quest been run before. More it has been run, more it has been battle tested.
          \item \textbf{Commitment }: Number of Learn Coins committed to this Quest - thereby suggesting the market value of the skill taught in this Quest. Resets to zero at every edition of the Quest.
          \item \textbf{Average time to complete Quest }: Those who have been awarded an NFT-L in an earlier edition of this Quest, how long did it take them to get it. Indication of time commitment required from the learner. 
          \item \textbf{Last issued NFT-L }:  When was the last NFT-L issued in any earlier or current edition of this Quest. This shows how active the Quest is right now. 
          \item \textbf{Pre-requisite NFT-L }: An NFT-L that is suggested pre-requisite to attend this Quest. Suggests what knowledge is required to make the most off of this Quest.
        \end{itemize}
    \subsection{Learning Material}
        The learning material is public and anyone can view it without participating in the Quest. 
        These are typically links, videos or PDFs. The user can see if they want to master the material by participating in the Quest. 
        Allowing learners to consume instructional material outside class, and using discussions to answer questions and drive curiousity has proven better results. %TODO(ruchil) : citation
        \par
        Learning Material is only a small part of the learning process. Lot of people are used to assimilating information but never applying it. 
        Anyone is free to open these learning material. 
        The ones who want to master can choose to participate in the Quest by requesting an entry to the Quest.
        Questbook focuses on Mastery. %TODO : JonB's diamond?
    \subsection{Requesting Entry}
        All the Quests are gated and number of participants limited. To participate, one must produce one of :
        \subsubsection{A pre-requisite NFT-L}
          If the learner owns the pre-requisite NFT-L they are given automatic entry to the Quest on a first come first serve basis. The NFT-L is automatically staked in the Quest.
        \subsubsection{Staking Learn Coins}
          If the learner doesn't have the pre-requisite NFT-L and wants to participate, they can stake some Learn Coins. This information is available to the Questmaster when they are selecting learners to allow.
    \subsection{Completing a Quest}
        \subsubsection{Public Chat}
          The Quest can be run in any format the Questmaster deems appropriate, by coordinating with the participants directly over a public chat. 
          Questbook recommends the Questmasters conduct all the communications in the public. Questbook automatically archives all the chats in a public repository.
          This public repository can be accessed whenever someone wants to audit a particular NFT-L issued in this Quest.
        \subsubsection{Request for NFT-L}
          The issuance of the NFT-L is initiated by the learner. The learner uploads a \textit{Request for NFT-L} as a special attachment in the Questroom chat.
        \subsubsection{Approval or Rejection of the NFT-L}
          The Questmaster can then look at the \textit{Request for NFT-L} and either accept it or reject it along with a voice/text that stays as an authentic commentary with the NFT-L forever, if approved. 
          If rejected, the chat continues and the learner can attempt a request any number of times.
        \subsubsection{Recovering the Stake}
          If the NFT-L is issued, the participants gets back, along with the new NFT-L, the staked NFT-L or the staked Learn Coins.
          \par
          If the participant is not able to get an NFT-L before the Quest ends, the staked NFT-L is burnt and the staked Learn Coins and NFT-Ls are burnt.
  \section{Voting process}
    Voting for new Quests and re-runs of existing Quests is a core pillar of this economy. 
    Voting happens by staking Learn Coins on a particular Quest.
    Voting by staking captures the following.
    \begin{itemize}
      \item \textbf{Determining what needs to be taught on the platform, and in the world.}
      \item Shows market demand for the learning outcomes - e.g. recruiters could vote to get access to potential hires,  NGOs can spread awareness.
      \item This is also how the Questmaster is incentivized monetarily. We don't want to charge the learners for learning.
    \end{itemize}
  \section{Uses of NFT-L}
    \subsection{Job Opportunities}
      The institutions who had paid the Questmaster for a particular Quest, could have had the motive of building skills that they want to hire. Producing such an NFT-L to such institutes will be a great optimization in their hiring process.
    \subsection{Grants and Scholarships}
      Questbook will be working closely with institutions and organizations to give out grants like Gitcoin Grants or Scholarships from governments or other NGOs.
  \section{Uses of a Voting by Staking}
    \subsection{Access to graduates}
      The more you commit to a Quest, the sooner you know about the \textit{graduates} of that Quest.
      The more the difference in your stake as against the next in line, the more the time difference between when you get access to the graduates as against the next. %TODO(madhavan) : equation
    \subsection{Recommending learners}
      By having commitment in the Quest, community can recommend few of their members to undergo the Quest to upskill themselves.
    \subsection{Access to audit}
      When the Quest is on-going, the chat history is not public. It is made public only when the Quest has ended.
      The users who have committed Learn Coins on a Quest, get access to chat history before the Quest has ended. 
      They get access to chat history as a function of Learn Coins committed.

  \section{Initial incentivization Model}
    To bootstrap the initial liquidity on the platfrom we will be incentivizing the early customers with Learn Coins - based on a definitive logic.
    First 3M Learn Coins will be distributed as a part of the Initial incentivization plan.
    \subsection{Questmasters initial Incentive Plan}
      Total pool = 2M Learn Coins.
      The factors that define the number of Learn Coins a Questmaster \(L\) gets on dropping the first Quest :
      \begin{itemize}
        \item Sum of \textbf{Number of Learn Coins in votes} across all Quests by that Questmaster, \textit{v}
        \item \textbf{Authority score} on other social media, \textit{s}
        \item \textbf{Editors' score} based on who will be most useful to the platform as decided by a jury, \textit{E}
      \end{itemize} 
      \[
        TODO
        %L = \sum v + (S * s) + E
      \]
      \(S = 0.002\)

    \subsection{Learners Initial Incentive Plan}
      Total Pool = 1M Learn Coins. The factors that define number of coins a learner gets upon completion of their first Quest \(L\)
      \begin{itemize}
        \item \textbf{Authority score} on other social media, \textit{s}
        \item \textbf{Time to acceptance} into the first applied Quest, \(t_{accept}\) (number of hours)
        \item \textbf{Time to complete Quest}, \(t_{complete}\) (number of hours)
        \item \textbf{Editor's score} based on who will be most useful to the platform as decided by a jury, \textit{E}
      \end{itemize}
      \[
        TODO
        %L = \frac{S * s}{10 * (5*t_{accept} + t_{complete})} + E
      \]
    \subsection{Ongoing incentivization (liquidity mining)}
      Every month 100,000 Learn Coins will be distributed to the top \(5\%\) of the Questmasters using the following weighting \(w\) 
      \subsubsection{Questmaster}
        \begin{itemize}
          \item Number of NFT-Ls issued ever by this Questmaster, \(N\)
          \item Number of Learn Coins paid as vote in the last completed Quest \(v\)
          \item Number of NFT-Ls issued by this Questmaster staked and recovered ever \(R\)
          \item Number of Learn Coins committed to last ended Quest \(\sigma\)
          \item Number of months since the ending of the last Quest \(t\)
        \end{itemize}
        Equation for distributing Learn Coins to Questmasters :
        \[
          w =  TODO
        \]

      \subsubsection{Learner }  
        Every month 100,000 Learn Coins will be distributed to the top \(20\%\) of the Learners using the following weighting \(w\) 
        \begin{itemize}
          \item NFT-Ls earned in Quest \(q\)
          \item Total Learn Coins used to vote on the Quest \(\sigma_q\)
          \item Total NFT-Ls issued so far in the Quest \(N_q\)
        \end{itemize}
        \[
          w = TODO
        \]
      \subsection{Community}
        Every month 100,000 Learn Coins will be distributed to the top \(5\%\) of the community members using the following weighting \(w\) 
        \begin{itemize}
          \item Amount committed across Quests \((C)\)
          \item Number of NFT-Ls issued by the Quests where the member committed \((N)\)
          \item Number of NFT-Ls staked and recovered issues by the Quests where the member committed \((R)\)
          \item Amount of time from making the commitment to the time the Quest \((q)\) started, \(t_q\)
        \end{itemize}
        \[
          w = TODO
        \]
        This allows the community to speculate on which Quests are most likely to be the \textit{best Quests}. Thereby, rewarding making early bets on promising Quests.
  \section{Governance and Future Work}
    A core fundamental of Questbook is to make learning free and accessible to everyone. 
    We think it is crucial for such a system to operate as a true decentralized autonomous organization. 
    A decentralized learning economy would be one in which the economic incentives are directly aligned with the learning outcomes on the platform.
    Though Questbook is the first to implement the specifications in this paper, it is crucial that Questbook is no longer a centralized entity deciding what gets taught and learnt in the world.
    \par
    To make it truly decentralized the following will be crucial.
    \subsection{Open Sourced}
      Questbook will be completely open sourced and anyone is free to run their own instance of Questbook and the LearnT protocol.
      Questbook itself will also be exposing APIs that can be used to build a thriving market of plugins on top of the platform itself, without having to fork Questbook.

    \subsection{Proof-of-stake}
      Questbook will transition into a Proof-of-stake consensus model for governance by staking Learn Coins.
      Here are a few examples of governance decisions 
      \begin{itemize}
        \item API specifications
        \item Incentive model updates
        \item Core product updates and merging pull requests
        \item Feed ranking algorithm 
      \end{itemize}

    Questbook is fully committed to building the world's first \textit{University DAO}.
    We fundamentally believe education should not be centralized and authoritarian.

  \section{Open Questions}
    The following are questions that are work in progress. If you think you can help us with any of the following please reach out to us at \texttt{madhavan@creatoros.co}
    \begin{itemize}
      \item How do you ensure teachers do not use commitments as a way to extract fees? A teacher could say things like "I will allow you into  this Quest only if you vote with 100 Learn Coins or more"
      \item The entire economy is being built to drive quality learning outcomes. How do we measure learning outcomes?
      \item Equations for liquidity mining
      \item Ensuring the correct order of access of graduates to committers. How do we ensure that Alice and Bob who have committed 100 Learn Coins and 50 Learn Coins respectively - get access to the chats and newly issued nft-ls only in the order defined by the equations. This becomes more relevant when the system is completely decentralized. The files to stay encrypted, and Bob gets access to the encrypted files only when it is his turn. None of the governance nodes or Charlie can see this information until it is their turn. This cryptography needs to be defined.
    \end{itemize}

  \begin{thebibliography}{9}
    \bibitem{starresearch} CLASS-SIZE POLICY: THE STAR EXPERIMENT AND RELATED CLASS-SIZE STUDIES, NCPEA	POLICY	BRIEF, October 2012
  \end{thebibliography}

\end{document}

